\documentclass[12pt,a4paper]{article}
\usepackage[]{color}
\usepackage[]{graphicx}
\usepackage[utf8x]{inputenc}
\usepackage{amsfonts}
\usepackage{amsmath}
\usepackage{amssymb}
\usepackage{eepic}
\usepackage{epic}
\usepackage{fullpage}
\usepackage{listings}
\usepackage{parskip}
\usepackage{ucs}
\usepackage{verbatim}

\title{
    {\normalsize\sc ID1217 Concurrent Programming} \\ 
    {\large Programming project} \\
    Parallelize Particle Simulation
}

\author{
    \bf Joel Pettersson \\
    \bf 880519-0637 \\
}

\lstset{%
basicstyle=\small,  % print whole listings small
}

\begin{document}

\maketitle

\section{Introduction}

My main objectives for this project was to better understand development of
parallel applications in both shared and distributed memory models and practise
improving parallel performance of an existing application.

The task was to parallelize a simplified particle simulator, where each
particle is only affected by other particles within a certain range, and where
the particle density is sufficiently low so that only $O(n)$ interactions are
expected, given $n$ particles.

\section{Method}

I started by thinking of a way to make the serial implementation run in $O(n)$
instead of $O(n^2)$ as was the case with the given code. The reason was that
each particle was checked for interaction with all other, despite the fact that
only particles in a certain range would affect it. 

I needed a way to quickly find only the particles that were candidates for
interacting with a specific other particle. What I came up with was an ADT that
I chose to call a \emph{Grid hash set}.

\subsection{Grid hash set}

The Grid hash set is made to hold references to all the particles. It works by
creating a grid of cells and each particle is put into one of these cells upon
insertion into the Grid hash set. When interaction candidates\footnote{
Particles that are possibly within another particle's range of interaction
force.} are asked for, only the surrounding cells whose areas are overlapped by
the interaction range of the given particle are searched. This is done in
constant amortized time for each particle, since the calculation of the grid
index (which cell to put the particle in) is constant, and
\texttt{vector::push\_back} is assumed to be constant in the long run. The
particles may be assumed to be spread evenly (on average) all over the grid,
and so there is only a constant amount of interaction candidates. Thus,
iterating over all particles, and for each one looking up a constant amount of
other particles, for which constant work is done, results in a linear time
complexity.


\subsubsection{Utilizing the Grid hash set}

When I had implemented and tested the Grid hash set it was only a matter of
utilizing it in the already existing applications. Generally, I simply started
the execution with inserting all particles into the grid and then distributed
the particles among the available workers.


\subsection{Synchronization shared memory implementations}

The synchronization in the Pthreads implementation simply consists of a single
barrier (used multiple times). This causes the simulation to be divided into
three disjoint phases; first the forces are computed for the particles
belonging to each thread, then the particles are moved, and finally the main
thread saves the new positions.

In the OpenMP implementation I use the \texttt{parallel} for the main thread to
create a team of threads to carry out the iteration indicated by \texttt{for}
in parallel. This essentially breaks down into the same behaviour as described
in previous paragraph, since there is an implied barrier at the end of the
\texttt{for} construct.


\subsection{Communication in distributed implementation}

For the sake of implementation simplicity I chose a type of communication that
made the parallelization construct resemble shared memory. To start with, the
main process initiates all particles and the broadcasts them to the rest of the
world. Then each process puts the particles into its own Grid hash set,
calculates the forces for its own particles (possibly based on all other
particles), moves them, and then when everyone is done, they send out their own
particles and gathers the new position of all other particles from the other
processes.


\section{Result}

The plot in Figure \ref{fig:linear} shows that all implementations run in
$O(n)$ time.

\begin{figure}[h]
    \begin{center}
        % GNUPLOT: LaTeX picture using EEPIC macros
\setlength{\unitlength}{0.120450pt}
\begin{picture}(3000,1800)(0,0)
\footnotesize
\color{black}
\color{black}
\thinlines \drawline[-50](431,265)(2876,265)
\color{black}
\thicklines \path(431,265)(472,265)
\thicklines \path(2876,265)(2835,265)
\put(390,265){\makebox(0,0)[r]{ 0.1}}
\color{black}
\thicklines \path(431,352)(451,352)
\thicklines \path(2876,352)(2856,352)
\thicklines \path(431,468)(451,468)
\thicklines \path(2876,468)(2856,468)
\thicklines \path(431,527)(451,527)
\thicklines \path(2876,527)(2856,527)
\color{black}
\thinlines \drawline[-50](431,555)(2876,555)
\color{black}
\thicklines \path(431,555)(472,555)
\thicklines \path(2876,555)(2835,555)
\put(390,555){\makebox(0,0)[r]{ 1}}
\color{black}
\thicklines \path(431,643)(451,643)
\thicklines \path(2876,643)(2856,643)
\thicklines \path(431,758)(451,758)
\thicklines \path(2876,758)(2856,758)
\thicklines \path(431,818)(451,818)
\thicklines \path(2876,818)(2856,818)
\color{black}
\thinlines \drawline[-50](431,846)(2876,846)
\color{black}
\thicklines \path(431,846)(472,846)
\thicklines \path(2876,846)(2835,846)
\put(390,846){\makebox(0,0)[r]{ 10}}
\color{black}
\thicklines \path(431,933)(451,933)
\thicklines \path(2876,933)(2856,933)
\thicklines \path(431,1049)(451,1049)
\thicklines \path(2876,1049)(2856,1049)
\thicklines \path(431,1108)(451,1108)
\thicklines \path(2876,1108)(2856,1108)
\color{black}
\thinlines \drawline[-50](431,1136)(2876,1136)
\color{black}
\thicklines \path(431,1136)(472,1136)
\thicklines \path(2876,1136)(2835,1136)
\put(390,1136){\makebox(0,0)[r]{ 100}}
\color{black}
\thicklines \path(431,1224)(451,1224)
\thicklines \path(2876,1224)(2856,1224)
\thicklines \path(431,1339)(451,1339)
\thicklines \path(2876,1339)(2856,1339)
\thicklines \path(431,1398)(451,1398)
\thicklines \path(2876,1398)(2856,1398)
\color{black}
\thinlines \drawline[-50](431,1427)(472,1427)
\thinlines \drawline[-50](1087,1427)(2876,1427)
\color{black}
\thicklines \path(431,1427)(472,1427)
\thicklines \path(2876,1427)(2835,1427)
\put(390,1427){\makebox(0,0)[r]{ 1000}}
\color{black}
\thicklines \path(431,1514)(451,1514)
\thicklines \path(2876,1514)(2856,1514)
\thicklines \path(431,1630)(451,1630)
\thicklines \path(2876,1630)(2856,1630)
\thicklines \path(431,1689)(451,1689)
\thicklines \path(2876,1689)(2856,1689)
\color{black}
\thinlines \drawline[-50](431,1717)(2876,1717)
\color{black}
\thicklines \path(431,1717)(472,1717)
\thicklines \path(2876,1717)(2835,1717)
\put(390,1717){\makebox(0,0)[r]{ 10000}}
\color{black}
\color{black}
\thinlines \drawline[-50](431,265)(431,1717)
\color{black}
\thicklines \path(431,265)(431,306)
\thicklines \path(431,1717)(431,1676)
\put(431,182){\makebox(0,0){ 100}}
\color{black}
\thicklines \path(615,265)(615,285)
\thicklines \path(615,1717)(615,1697)
\thicklines \path(723,265)(723,285)
\thicklines \path(723,1717)(723,1697)
\thicklines \path(799,265)(799,285)
\thicklines \path(799,1717)(799,1697)
\thicklines \path(858,265)(858,285)
\thicklines \path(858,1717)(858,1697)
\thicklines \path(907,265)(907,285)
\thicklines \path(907,1717)(907,1697)
\thicklines \path(948,265)(948,285)
\thicklines \path(948,1717)(948,1697)
\thicklines \path(983,265)(983,285)
\thicklines \path(983,1717)(983,1697)
\thicklines \path(1014,265)(1014,285)
\thicklines \path(1014,1717)(1014,1697)
\color{black}
\thinlines \drawline[-50](1042,265)(1042,1344)
\thinlines \drawline[-50](1042,1676)(1042,1717)
\color{black}
\thicklines \path(1042,265)(1042,306)
\thicklines \path(1042,1717)(1042,1676)
\put(1042,182){\makebox(0,0){ 1000}}
\color{black}
\thicklines \path(1226,265)(1226,285)
\thicklines \path(1226,1717)(1226,1697)
\thicklines \path(1334,265)(1334,285)
\thicklines \path(1334,1717)(1334,1697)
\thicklines \path(1410,265)(1410,285)
\thicklines \path(1410,1717)(1410,1697)
\thicklines \path(1469,265)(1469,285)
\thicklines \path(1469,1717)(1469,1697)
\thicklines \path(1518,265)(1518,285)
\thicklines \path(1518,1717)(1518,1697)
\thicklines \path(1559,265)(1559,285)
\thicklines \path(1559,1717)(1559,1697)
\thicklines \path(1594,265)(1594,285)
\thicklines \path(1594,1717)(1594,1697)
\thicklines \path(1626,265)(1626,285)
\thicklines \path(1626,1717)(1626,1697)
\color{black}
\thinlines \drawline[-50](1654,265)(1654,1717)
\color{black}
\thicklines \path(1654,265)(1654,306)
\thicklines \path(1654,1717)(1654,1676)
\put(1654,182){\makebox(0,0){ 10000}}
\color{black}
\thicklines \path(1838,265)(1838,285)
\thicklines \path(1838,1717)(1838,1697)
\thicklines \path(1945,265)(1945,285)
\thicklines \path(1945,1717)(1945,1697)
\thicklines \path(2022,265)(2022,285)
\thicklines \path(2022,1717)(2022,1697)
\thicklines \path(2081,265)(2081,285)
\thicklines \path(2081,1717)(2081,1697)
\thicklines \path(2129,265)(2129,285)
\thicklines \path(2129,1717)(2129,1697)
\thicklines \path(2170,265)(2170,285)
\thicklines \path(2170,1717)(2170,1697)
\thicklines \path(2206,265)(2206,285)
\thicklines \path(2206,1717)(2206,1697)
\thicklines \path(2237,265)(2237,285)
\thicklines \path(2237,1717)(2237,1697)
\color{black}
\thinlines \drawline[-50](2265,265)(2265,1717)
\color{black}
\thicklines \path(2265,265)(2265,306)
\thicklines \path(2265,1717)(2265,1676)
\put(2265,182){\makebox(0,0){ 100000}}
\color{black}
\thicklines \path(2449,265)(2449,285)
\thicklines \path(2449,1717)(2449,1697)
\thicklines \path(2556,265)(2556,285)
\thicklines \path(2556,1717)(2556,1697)
\thicklines \path(2633,265)(2633,285)
\thicklines \path(2633,1717)(2633,1697)
\thicklines \path(2692,265)(2692,285)
\thicklines \path(2692,1717)(2692,1697)
\thicklines \path(2740,265)(2740,285)
\thicklines \path(2740,1717)(2740,1697)
\thicklines \path(2781,265)(2781,285)
\thicklines \path(2781,1717)(2781,1697)
\thicklines \path(2817,265)(2817,285)
\thicklines \path(2817,1717)(2817,1697)
\thicklines \path(2848,265)(2848,285)
\thicklines \path(2848,1717)(2848,1697)
\color{black}
\thinlines \drawline[-50](2876,265)(2876,1717)
\color{black}
\thicklines \path(2876,265)(2876,306)
\thicklines \path(2876,1717)(2876,1676)
\put(2876,182){\makebox(0,0){ 1e+06}}
\color{black}
\color{black}
\thicklines \path(431,1717)(431,265)(2876,265)(2876,1717)(431,1717)
\color{black}
\put(61,991){\makebox(0,0)[l]{\rotatebox[origin=c]{90}{Time (s)\hspace{10 mm}}}}
\color{black}
\color{black}
\put(1653,58){\makebox(0,0){Number of particles (n)}}
\color{black}
\color{black}
\color{red}
\color{black}
\put(800,1635){\makebox(0,0)[r]{Serial}}
\color{red}
\put(431,525){\makebox(0,0){$\Diamond$}}
\put(615,608){\makebox(0,0){$\Diamond$}}
\put(799,705){\makebox(0,0){$\Diamond$}}
\put(983,791){\makebox(0,0){$\Diamond$}}
\put(1167,882){\makebox(0,0){$\Diamond$}}
\put(1351,973){\makebox(0,0){$\Diamond$}}
\put(1535,1068){\makebox(0,0){$\Diamond$}}
\put(1719,1162){\makebox(0,0){$\Diamond$}}
\put(1903,1258){\makebox(0,0){$\Diamond$}}
\put(2087,1348){\makebox(0,0){$\Diamond$}}
\put(2271,1438){\makebox(0,0){$\Diamond$}}
\put(2455,1527){\makebox(0,0){$\Diamond$}}
\put(2639,1616){\makebox(0,0){$\Diamond$}}
\put(943,1635){\makebox(0,0){$\Diamond$}}
\color{blue}
\color{black}
\put(800,1552){\makebox(0,0)[r]{Pthreads}}
\color{blue}
\put(431,449){\makebox(0,0){$+$}}
\put(615,531){\makebox(0,0){$+$}}
\put(799,627){\makebox(0,0){$+$}}
\put(983,710){\makebox(0,0){$+$}}
\put(1167,803){\makebox(0,0){$+$}}
\put(1351,893){\makebox(0,0){$+$}}
\put(1535,989){\makebox(0,0){$+$}}
\put(1719,1084){\makebox(0,0){$+$}}
\put(1903,1179){\makebox(0,0){$+$}}
\put(2087,1271){\makebox(0,0){$+$}}
\put(2271,1361){\makebox(0,0){$+$}}
\put(2455,1450){\makebox(0,0){$+$}}
\put(2639,1540){\makebox(0,0){$+$}}
\put(2823,1631){\makebox(0,0){$+$}}
\put(943,1552){\makebox(0,0){$+$}}
\color{green}
\color{black}
\put(800,1469){\makebox(0,0)[r]{OpenMP}}
\color{green}
\put(431,448){\makebox(0,0){$\Box$}}
\put(615,529){\makebox(0,0){$\Box$}}
\put(799,624){\makebox(0,0){$\Box$}}
\put(983,710){\makebox(0,0){$\Box$}}
\put(1167,802){\makebox(0,0){$\Box$}}
\put(1351,893){\makebox(0,0){$\Box$}}
\put(1535,990){\makebox(0,0){$\Box$}}
\put(1719,1085){\makebox(0,0){$\Box$}}
\put(1903,1180){\makebox(0,0){$\Box$}}
\put(2087,1270){\makebox(0,0){$\Box$}}
\put(2271,1361){\makebox(0,0){$\Box$}}
\put(2455,1450){\makebox(0,0){$\Box$}}
\put(2639,1540){\makebox(0,0){$\Box$}}
\put(943,1469){\makebox(0,0){$\Box$}}
\color{magenta}
\color{black}
\put(800,1386){\makebox(0,0)[r]{MPI}}
\color{magenta}
\put(431,420){\makebox(0,0){$\times$}}
\put(615,505){\makebox(0,0){$\times$}}
\put(799,602){\makebox(0,0){$\times$}}
\put(983,688){\makebox(0,0){$\times$}}
\put(1167,780){\makebox(0,0){$\times$}}
\put(1351,867){\makebox(0,0){$\times$}}
\put(1535,956){\makebox(0,0){$\times$}}
\put(1719,1044){\makebox(0,0){$\times$}}
\put(1903,1143){\makebox(0,0){$\times$}}
\put(2087,1255){\makebox(0,0){$\times$}}
\put(2271,1357){\makebox(0,0){$\times$}}
\put(2455,1452){\makebox(0,0){$\times$}}
\put(2639,1544){\makebox(0,0){$\times$}}
\put(943,1386){\makebox(0,0){$\times$}}
\color{black}
\thicklines \path(431,1717)(431,265)(2876,265)(2876,1717)(431,1717)
\color{black}
\end{picture}

        \caption{Plot in log-log scale that indicates $\mathcal{O}(n)$ run time for
        both the serial and parallel implementations. See
        Appendix~\ref{sec:appendixBenchmarkScripts} for how the data was collected.} 
        \label{fig:linear}
    \end{center}
\end{figure}

The plot in Figure \ref{fig:speedup} indicates the parallel speedup, which does
not really reach the idealized $p$-times speedup.

\begin{figure}[h]
    \begin{center}
        % GNUPLOT: LaTeX picture using EEPIC macros
\setlength{\unitlength}{0.120450pt}
\begin{picture}(3000,1800)(0,0)
\footnotesize
\color{black}
\color{black}
\thinlines \drawline[-50](308,265)(2876,265)
\color{black}
\thicklines \path(308,265)(349,265)
\thicklines \path(2876,265)(2835,265)
\put(267,265){\makebox(0,0)[r]{ 0}}
\color{black}
\color{black}
\thinlines \drawline[-50](308,447)(2876,447)
\color{black}
\thicklines \path(308,447)(349,447)
\thicklines \path(2876,447)(2835,447)
\put(267,447){\makebox(0,0)[r]{ 10}}
\color{black}
\color{black}
\thinlines \drawline[-50](308,628)(2876,628)
\color{black}
\thicklines \path(308,628)(349,628)
\thicklines \path(2876,628)(2835,628)
\put(267,628){\makebox(0,0)[r]{ 20}}
\color{black}
\color{black}
\thinlines \drawline[-50](308,810)(2876,810)
\color{black}
\thicklines \path(308,810)(349,810)
\thicklines \path(2876,810)(2835,810)
\put(267,810){\makebox(0,0)[r]{ 30}}
\color{black}
\color{black}
\thinlines \drawline[-50](308,991)(2876,991)
\color{black}
\thicklines \path(308,991)(349,991)
\thicklines \path(2876,991)(2835,991)
\put(267,991){\makebox(0,0)[r]{ 40}}
\color{black}
\color{black}
\thinlines \drawline[-50](308,1173)(2876,1173)
\color{black}
\thicklines \path(308,1173)(349,1173)
\thicklines \path(2876,1173)(2835,1173)
\put(267,1173){\makebox(0,0)[r]{ 50}}
\color{black}
\color{black}
\thinlines \drawline[-50](308,1354)(2876,1354)
\color{black}
\thicklines \path(308,1354)(349,1354)
\thicklines \path(2876,1354)(2835,1354)
\put(267,1354){\makebox(0,0)[r]{ 60}}
\color{black}
\color{black}
\thinlines \drawline[-50](308,1536)(2220,1536)
\thinlines \drawline[-50](2835,1536)(2876,1536)
\color{black}
\thicklines \path(308,1536)(349,1536)
\thicklines \path(2876,1536)(2835,1536)
\put(267,1536){\makebox(0,0)[r]{ 70}}
\color{black}
\color{black}
\thinlines \drawline[-50](308,1717)(2876,1717)
\color{black}
\thicklines \path(308,1717)(349,1717)
\thicklines \path(2876,1717)(2835,1717)
\put(267,1717){\makebox(0,0)[r]{ 80}}
\color{black}
\color{black}
\thinlines \drawline[-50](308,265)(308,1717)
\color{black}
\thicklines \path(308,265)(308,306)
\thicklines \path(308,1717)(308,1676)
\put(308,182){\makebox(0,0){ 1}}
\color{black}
\color{black}
\thinlines \drawline[-50](675,265)(675,1717)
\color{black}
\thicklines \path(675,265)(675,306)
\thicklines \path(675,1717)(675,1676)
\put(675,182){\makebox(0,0){ 2}}
\color{black}
\color{black}
\thinlines \drawline[-50](1042,265)(1042,1717)
\color{black}
\thicklines \path(1042,265)(1042,306)
\thicklines \path(1042,1717)(1042,1676)
\put(1042,182){\makebox(0,0){ 3}}
\color{black}
\color{black}
\thinlines \drawline[-50](1409,265)(1409,1717)
\color{black}
\thicklines \path(1409,265)(1409,306)
\thicklines \path(1409,1717)(1409,1676)
\put(1409,182){\makebox(0,0){ 4}}
\color{black}
\color{black}
\thinlines \drawline[-50](1775,265)(1775,1717)
\color{black}
\thicklines \path(1775,265)(1775,306)
\thicklines \path(1775,1717)(1775,1676)
\put(1775,182){\makebox(0,0){ 5}}
\color{black}
\color{black}
\thinlines \drawline[-50](2142,265)(2142,1717)
\color{black}
\thicklines \path(2142,265)(2142,306)
\thicklines \path(2142,1717)(2142,1676)
\put(2142,182){\makebox(0,0){ 6}}
\color{black}
\color{black}
\thinlines \drawline[-50](2509,265)(2509,1427)
\thinlines \drawline[-50](2509,1676)(2509,1717)
\color{black}
\thicklines \path(2509,265)(2509,306)
\thicklines \path(2509,1717)(2509,1676)
\put(2509,182){\makebox(0,0){ 7}}
\color{black}
\color{black}
\thinlines \drawline[-50](2876,265)(2876,1717)
\color{black}
\thicklines \path(2876,265)(2876,306)
\thicklines \path(2876,1717)(2876,1676)
\put(2876,182){\makebox(0,0){ 8}}
\color{black}
\color{black}
\thicklines \path(308,1717)(308,265)(2876,265)(2876,1717)(308,1717)
\color{black}
\put(61,991){\makebox(0,0)[l]{\shortstack{T\\i\\m\\e\\ \\i\\n\\ \\s\\e\\c\\o\\n\\d\\s}}}
\color{black}
\color{black}
\put(1592,58){\makebox(0,0){Number of workers}}
\color{black}
\color{black}
\color{red}
\color{black}
\put(2548,1635){\makebox(0,0)[r]{Pthreads}}
\color{red}
\put(308,1555){\makebox(0,0){$\Diamond$}}
\put(675,1003){\makebox(0,0){$\Diamond$}}
\put(1042,797){\makebox(0,0){$\Diamond$}}
\put(1409,692){\makebox(0,0){$\Diamond$}}
\put(1775,627){\makebox(0,0){$\Diamond$}}
\put(2142,582){\makebox(0,0){$\Diamond$}}
\put(2509,550){\makebox(0,0){$\Diamond$}}
\put(2876,517){\makebox(0,0){$\Diamond$}}
\put(2691,1635){\makebox(0,0){$\Diamond$}}
\color{blue}
\color{black}
\put(2548,1552){\makebox(0,0)[r]{OpenMP}}
\color{blue}
\put(308,1543){\makebox(0,0){$+$}}
\put(675,999){\makebox(0,0){$+$}}
\put(1042,794){\makebox(0,0){$+$}}
\put(1409,689){\makebox(0,0){$+$}}
\put(1775,624){\makebox(0,0){$+$}}
\put(2142,580){\makebox(0,0){$+$}}
\put(2509,542){\makebox(0,0){$+$}}
\put(2876,515){\makebox(0,0){$+$}}
\put(2691,1552){\makebox(0,0){$+$}}
\color{green}
\color{black}
\put(2548,1469){\makebox(0,0)[r]{MPI}}
\color{green}
\put(308,1574){\makebox(0,0){$\Box$}}
\put(675,950){\makebox(0,0){$\Box$}}
\put(1042,743){\makebox(0,0){$\Box$}}
\put(1409,640){\makebox(0,0){$\Box$}}
\put(1775,585){\makebox(0,0){$\Box$}}
\put(2142,552){\makebox(0,0){$\Box$}}
\put(2509,517){\makebox(0,0){$\Box$}}
\put(2876,515){\makebox(0,0){$\Box$}}
\put(2691,1469){\makebox(0,0){$\Box$}}
\color{black}
\thicklines \path(308,1717)(308,265)(2876,265)(2876,1717)(308,1717)
\color{black}
\end{picture}

        \caption{Plot that shows the speedup from parallelization. See
        Appendix~\ref{sec:appendixBenchmarkScripts} for how the data was
        collected.} 
        \label{fig:speedup}
    \end{center}
\end{figure}


% var spenderas mest tid? utgå från lite slarvig profilering
% hur skalar den tiden med antalet workers?


\section{Discussion}

% hur designvalen (gridhashset, allgatherv, \dots) påverkade presetnadn

% hur är det att använda pthreads, kontra OpenMP, kontra MPI?

\appendix

\section{Benchmark scripts}
\label{sec:appendixBenchmarkScripts}

\subsection{Linearity}
\lstinputlisting[language=bash]{../code/benchmarking/linear.sh}

\subsection{Parallelization}
\lstinputlisting[language=bash]{../code/benchmarking/speedup.sh}


\section{Benchmarks data} 

\subsection{Linearity}

\verbatiminput{../benchmarks/linear/serial.dat}
\verbatiminput{../benchmarks/linear/pthreads.dat}
\verbatiminput{../benchmarks/linear/openmp.dat}
\verbatiminput{../benchmarks/linear/mpi.dat}

\subsection{Parallelization}

\verbatiminput{../benchmarks/speedup/pthreads.dat}
\verbatiminput{../benchmarks/speedup/openmp.dat}
\verbatiminput{../benchmarks/speedup/mpi.dat}


\end{document}
